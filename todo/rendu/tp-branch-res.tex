% vim:set spell:
% vim:spell spelllang=fr:
\documentclass[a4paper]{article}
\usepackage[utf8x]{inputenc}
\usepackage[T1]{fontenc}
\usepackage{charter}
\usepackage{graphicx}
\usepackage{amsmath,amssymb}
\usepackage[french]{babel}
\usepackage{xspace}
\usepackage{setspace}
\setstretch{1.0}
\usepackage{subfigure}
\usepackage{listings}
\voffset       -1in
\hoffset       -1in
\headheight     12pt
\headsep        12pt
\topmargin      25mm
\oddsidemargin  20mm
\textwidth      170mm
\textheight     240mm
\flushbottom
\lstset{numbers=left, numberstyle=\tiny, stepnumber=1, numbersep=5pt}
\graphicspath{{../../scripts/}}
\begin{document}
\begin{center}
\large
Travaux Pratiques Archi SEOC-3A\\
\LARGE
Prédiction de branchements\\
\large

\end{center}
\section{Identification}
Travail réalisé par Frédéric Pétrot

\section{Prédicteur $n$ modal : conception et résultats}
\subsection{Conception}
Le prédicteur $n$ modal bit est constitué d'un unique tableau contenant de $2^1$ à $2^3$ bits.

\subsection{Résultats}
Les résultats issus de la simulation sont les suivants.
\par
par defaut

\begin{minipage}{.48\linewidth}
\includegraphics[width=\linewidth]{graph_type0_1}
\end{minipage}%
\hfill
\begin{minipage}{.48\linewidth}
\includegraphics[width=\linewidth]{graph_type0_2}
\end{minipage}
\begin{minipage}{.48\linewidth}
\includegraphics[width=\linewidth]{graph_type0_3}
\end{minipage}%
\hfill
\begin{minipage}{.48\linewidth}
\includegraphics[width=\linewidth]{graph_type0_4}
\end{minipage}

gshare

\begin{minipage}{.48\linewidth}
\includegraphics[width=\linewidth]{graph_type1_1}
\end{minipage}%
\hfill
\begin{minipage}{.48\linewidth}
\includegraphics[width=\linewidth]{graph_type1_2}
\end{minipage}
\begin{minipage}{.48\linewidth}
\includegraphics[width=\linewidth]{graph_type1_3}
\end{minipage}%
\hfill
\begin{minipage}{.48\linewidth}
\includegraphics[width=\linewidth]{graph_type1_4}
\end{minipage}
local

\begin{minipage}{.48\linewidth}
\includegraphics[width=\linewidth]{graph_type2_1}
\end{minipage}%
\hfill
\begin{minipage}{.48\linewidth}
\includegraphics[width=\linewidth]{graph_type2_2}
\end{minipage}
\begin{minipage}{.48\linewidth}
\includegraphics[width=\linewidth]{graph_type2_3}
\end{minipage}%
\hfill
\begin{minipage}{.48\linewidth}
\includegraphics[width=\linewidth]{graph_type2_4}
\end{minipage}
combined

\begin{minipage}{.48\linewidth}
\includegraphics[width=\linewidth]{graph_type3_1}
\end{minipage}%
\hfill
\begin{minipage}{.48\linewidth}
\includegraphics[width=\linewidth]{graph_type3_2}
\end{minipage}

\begin{minipage}{.48\linewidth}
\includegraphics[width=\linewidth]{graph_type3_3}
\end{minipage}%
\hfill
\begin{minipage}{.48\linewidth}
\includegraphics[width=\linewidth]{graph_type3_4}
\end{minipage}



\subsection{Analyse}
\subsubsection{Prédicteur par défaut}
On voit une asymptote due à la disparition des collisions lorsque la taille du prédicteur augmente.
Le coût du prédicteur est linéaire avec la taille du tableau, et il n'est pas raisonnable de dépasser $2^{16}$ éléments, d'autant que le gain à partir de $2^{12}$ devient très faible.
Par ailleurs, il y a toujours moins de $7\%$ de mauvaise prédictions, ce qui est remarquable pour une approche aussi simpliste.

\subsubsection{Prédicteur gshare}

On observe une asymptote claire lorsque la taille de la table augmente
ce qui s'explique par la réduction progressive des collisions d'indexation entre branches partageant la même entrée.
Le coût du prédicteur étant linéaire avec la taille de la table, il n'est pas pertinent de dépasser une taille de l'ordre de $2^{14}$ entrées.
d’autant plus que le gain devient très faible au-delà de $2^{12}$
Grâce à l’utilisation de l’historique global, le prédicteur gshare capture efficacement les corrélations entre branches.
Ce qui lui permet de maintenir un taux de mauvaises prédictions faible, généralement inférieur à quelques pourcents, même avec des ressources matérielles limitées.

\subsubsection{Prédicteur local}
Comme pour gshare, une asymptote apparaît lorsque la taille du prédicteur augmente traduisant la diminution des collisions dans la table.
Toutefois, la convergence est plus lente et le taux de mauvaises prédictions reste globalement plus élevé.
Le coût matériel augmentant linéairement avec la taille du prédicteur.
Le gain devient marginal au-delà de $2^{12}$.
Le prédicteur local a un comportement stable pour les branches régulières.

\subsubsection{Prédicteur combiné}
Le prédicteur combined présente également une asymptote lorsque la taille de la table augmente, 
mais celle-ci est atteinte plus rapidement et à un niveau de mauvaises prédictions plus faible.
Le coût du prédicteur restant linéaire avec la taille des tables et du méta-Prédicteur
Le gain devient marginal au-delà de $2^{12}$.
cette approche permet de s'adapter dynamiquement au comportement des branches et d'obtenir systématiquement le meilleur taux de prédiction, souvent inférieur à celui de chacun des prédicteurs pris isolément, ce qui en fait une solution particulièrement efficace et robuste.
\end{document}
